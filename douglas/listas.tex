\documentclass[11pt,a4paper]{article}
\usepackage{estilosexercicios}
\usepackage{hyperref}
%https://yutsumura.com
%https://yutsumura.com/polynomial-ring-with-integer-coefficients-and-the-prime-ideal-ifx-in-zx-mid-f-20/
%\usepackage[bottom=2cm,top=3cm,left=3cm,right=2cm]{geometry}
%\usepackage[utf8]{inputenc}
%Environments para esta lista
% ---------------------------------------------------
\definecolor{Floresta}{rgb}{0.13,0.54,0.13}
\newcommand{\exercicio}[1]{\subsection{Exercício #1} \textcolor{blue}{\bf(#1)}}
\newcommand{\dividiritens}[1]{\begin{tasks}[counter-format={(tsk[a])},label-width=3.6ex, label-format = {\bfseries}, column-sep = {0pt}](1) #1 \end{tasks}}
\newcommand{\pers}[1]{\textcolor{Floresta}{$\negrito{(#1)} $}}

\newcommand{\solucao}[1]{
\textbf{\textcolor{white}{oi}\\ \\ \textcolor{red}{Solução:}} #1}
\newcommand{\figura}[1]{\input Arquivos_de_figs_Exercicios/#1} %Adicionar figuras do latex

% ---------------------------------------------------
\title{Tópicos de Anéis e Módulos}
\author{MAT0501/MAT6680}
\date{2º semestre de 2019}

\begin{document}
\definecolor{Floresta}{rgb}{0.13,0.54,0.13}
\maketitle
\tableofcontents
\newpage
\begin{comment}

\begin{center}
\large\textbf{\textcolor{Floresta}{Lista 1}}\\
\end{center}

\end{comment}

\section{\textcolor{Floresta}{Lista 1}}

\exercicio{1} Seja $A$ um conjunto com duas operações que satisfazem todas as condições da definição de anel com
unidade, com a possível exceção da condição: $a + b = b + a$ para todo $a, b \in A.$ Prove que $A$ é um anel.
\solucao{
Seja $(A,+,0,-,\cdot,1)$ uma estrutura tal que:
\begin{itemize}
\item[A1)] $x+(y+z)=(x+y)+z$,
\item[A2)] $x+0=0+x=x$,
\item[A3)] $x+(-x)=(-x)+x=0$,
\item[M1)] $x\cdot(y\cdot z)=(x\cdot y)\cdot z$,
\item[D1)] $x\cdot(y+z)=(x\cdot y)+(x\cdot z)$,
\item[D2)] $(x+y)\cdot z=(x\cdot z)+(y\cdot z)$,
\item[M2)] $x\cdot 1=1\cdot x=x$.
\end{itemize}

Vamos calcular o valor de $(x+y)(1+1):$
Por um lado, temos:
\[
\begin{array}{rcl}
(x+y)\cdot \textcolor{Purple}{(1+1)}&=& x\cdot\textcolor{Purple}{(1+1)}+y\cdot \textcolor{Purple}{(1+1)}\\&=&x\cdot \textcolor{Purple}{1}+x\cdot \textcolor{Purple}{1}+y\cdot \textcolor{Purple}{1}+y\cdot \textcolor{Purple}{1}\\&=&x+x+y+y,
\end{array}
\]
mas também:
\[
\begin{array}{rcl}
\textcolor{Brown}{(x+y)}\cdot(1+1)&=& \textcolor{Brown}{(x+y)}\cdot 1+\textcolor{Brown}{(x+y)}\cdot 1\\&=&x+y+x+y,
\end{array}
\]
Dessa forma:
\[
(x+y)(1+1) = (x+y)(1+1) \Rightarrow \cancel{x} + x + y + \bcancel{y} = \cancel{x} + y + x + \bcancel{y} \Rightarrow x + y = y + x
\]
\[
x+x+y+y=x+y+x+y \Rightarrow \boxed{x+y=y+x}
\]
Concluímos portanto que $x+y=y+x.$ Logo, combinado com as demais condições, $A$ é um anel.


}
\exercicio{2}
Seja $A$ um anel e $R$ um subanel de $A.$ Pode acontecer que:
\dividiritens{
\task[\pers{a}] $A$ seja um anel com unidade e $R$ não;
\task[\pers{b}] $R$ seja um anel com unidade e $A$ não;
\task[\pers{c}] $A$ e $R$ sejam anéis com unidade e a unidade de $A$ seja diferente da unidade de $R;$
\task[\pers{d}] $A$ e $R$ sejam anéis com unidade e as unidades de $A$ e de $R$ coincidem.
}
Dar exemplos que ilustrem cada uma das situações acima.

\solucao{
  \dividiritens{
\task[\pers{a}] Considere $A = \mathbb{Z}$ e $R = 2 \mathbb{Z}.$ Observe que $ \mathbb{Z}$ é um anel com unidade, mas $2 \mathbb{Z}$ não.

\task[\pers{b}] Considere $R = \mathbb{Z} \times \{ 0 \}.$ Então $R$ é um anel com unidade. $R$ é subanel de $A = \mathbb{Z} \times 2 \mathbb{Z},$ que é um anel que não possui unidade.
}
\dividiritens{
\task[\pers{c}] Temos vários exemplos interessantes. Citamos $3$ aqui:
\begin{itemize}
    \item Sejam
    \[
A:=\left\{\;\begin{pmatrix}x&y\\z&w\end{pmatrix}\;;\;\;x,y,z,w\in\Bbb R\;\right\}\; e \;R:=\left\{\;\begin{pmatrix}a&a\\a&a\end{pmatrix}\;;\;\;a\in\Bbb R\;\right\}
    \]
    $R$ é subanel de $A,$ e 
    \[
    1_A = \begin{pmatrix}1&0\\0&1\end{pmatrix} \neq 1_R = \begin{pmatrix}\frac{1}{2}&\frac{1}{2}\\\frac{1}{2}&\frac{1}{2}\end{pmatrix}
    \]
    \end{itemize}
    \begin{itemize}
        \item Sejam
    \[
A:=\left\{\;\begin{pmatrix}x&y\\z&w\end{pmatrix}\;;\;\;x,y,z,w\in\Bbb R\;\right\}\; e \;R:=\left\{\;\begin{pmatrix}a&0\\0&0\end{pmatrix}\;;\;\;a\in\Bbb R\;\right\}
    \]
    $R$ é subanel de $A,$ e 
    \[
    1_A = \begin{pmatrix}1&0\\0&1\end{pmatrix} \neq 1_R = \begin{pmatrix}1&0\\0&0\end{pmatrix}
    \]
    \end{itemize}

\task[\pers{d}]
}


}
    \begin{itemize}

            \item Considere o grupo diedral de ordem $8:$
            \[
            D_4 = \langle \sigma, \tau | \sigma^4 = \tau^2 = 1, \tau \sigma = \sigma^3 \tau \rangle
            \]
Considere o anel de grupo $A = \mathbb{C}[D_4],$ um espaço vetorial de dimensão $8$ sobre $\mathbb{C}$ com base $\{e_{1}, e_{\sigma}, e_{\sigma^2}, e_{\sigma^3}, e_{\tau}, e_{\sigma \tau}, e_{\sigma^2\tau}, e_{\sigma^3 \tau} \}.$ A multiplicação é dada por $e_ge_h = e_{gh}.$ Logo, a unidade de $A$ é $e_{1}.$

Como o centro de $D_4$ é $\{1, \sigma^2 \},$ $v_1 = \frac{1}{2}(e_1 - e_{\sigma^2})$ comuta com todos os elementos de $A.$ Além disso, $v_1$ é idempotente. Para cada $g \in D_4,$ defina $v_g = V_1 e_g.$ Pode-se verificar que $v_g v_h = v_{gh},$ e que $v_{\sigma^2} = -v_1,$ assim
\[
\{ v_g |g \in D_4 \} = \{\pm v_1, \pm v_\sigma, \pm v_\tau, \pm v_\sigma v_\tau \}
\]
tomemos então
\[
R = \{ a_v1 + bv_\sigma + cv_\tau + d v_{\sigma \tau} |a,b,c,d \in \mathbb{C} \}
\]
$R$ é um subconjunto de $A$ fechado por subtração e multiplicação, e portanto um subanel de $A,$ cujo elemento identidade é $v_1.$ Assim,
\[
1_A = e_1 \neq v_1 = 1_R
\]
\end{itemize}

\exercicio{3} Prove que o único automorfismo do anel $\mathbb{Z}$ é o automorfismo idêntico.
\solucao{
Seja $\varphi \colon \mathbb{Z} \to \mathbb{Z}$ um automorfismo de $\mathbb{Z}.$ Note primeiramente que
\[
\varphi(1) = \varphi(1 \cdot 1) = \varphi(1) \cdot \varphi(1) \Rightarrow \varphi(1) = (\varphi(1))^2
\]
Como os únicos elementos idempotentes de $\mathbb{Z}$ são o $0$ e o $1,$ e em particular $\varphi$ é um isomorfismo, temos que a única possibilidade é $\varphi(1) = 1.$ Assim, para todo $z \in \mathbb{Z}^{+},$ temos que $z = n = \underbrace{1 + 1 + \ldots + 1}_{n \mbox{ vezes}}.$ Daí
\[
\varphi(z) = \varphi(n) = \varphi(1 + 1 + \ldots + 1) = n \varphi(1)
\]
Assim, $\varphi(z) = z \ \forall z \in \mathbb{Z}^{+}.$ Para $z \in \mathbb{Z}^{-},$ temos que $-z \in \mathbb{Z}^{+},$ e então
\[
\varphi(z) = -\varphi(-z) = -(-z) = z
\]
Portanto, temos que $\varphi(z) = z$ para todo $z \in \mathbb{Z},$ ou seja, trata-se do automorfismo idêntico.
}

\exercicio{4} Sejam $A$ um corpo, $A^{\prime}$ um anel e $\varphi \colon A \to A^{\prime}$ um homomorfismo de anéis não nulo. Prove que $\varphi$ é monomorfismo.
\solucao{
Suponha por absurdo que $\varphi$ não seja um monomorfismo. Dessa forma, existem $x, y \in A$ tais que $\varphi(x) = \varphi(y),$ mas $x \neq y.$ Observe que, como $\varphi$ é homomorfismo de anéis, temos:
\[
\varphi(x) = \varphi(y) \Rightarrow \varphi(x) - \varphi(y) = 0 \Rightarrow \varphi(x - y) = 0
\]
Sendo $A$ um corpo, como $x - y \in A,$ e $x - y \neq 0,$ este admite um inverso, digamos $(x-y)^{-1}.$ Assim, temos que
\[
\varphi(x-y)\varphi((x-y)^{-1}) = 0 \Rightarrow \varphi((x-y)(x-y)^{-1}) = 0 \Rightarrow \varphi(1) = 0
\]
Assim, para todo $r \in A,$ temos que
\[
\varphi(r) = \varphi(1 \cdot r) = \textcolor{red}{\varphi(1)} \varphi(r) = \textcolor{red}{0} \varphi(r) = 0
\]
Portanto, $\varphi$ seria o homomorfismo nulo, uma contradição. Logo,  $\varphi$ é injetora.

Outra solução: Um homomorfismo de anéis é injetor se e somente se $\ker \varphi = \{ 0 \}.$ Sabemos também que $\ker \varphi$ é um ideal de $A.$ Sendo $A$ corpo, seus únicos ideais são os triviais. Então $\ker \varphi = \{ 0 \}$ ou $\ker \varphi = A.$ No primeiro caso, concluímos que $\varphi$ é injetora. Na segunda, $\varphi$ corresponde ao homomorfismo nulo, o que não é o caso. Logo, $\varphi$ é injetora.

Cabe salientar que o resultado não é válido se $A$ for apenas um anel. Por exemplo, tomando
\[
\fullfunction{\phi}{\mathbb{Z}}{\mathbb{Z}_2}{z}{\overline{z}},
\]
temos um homomorfismo de anéis que não é nem o nulo e nem injetor.
}

\exercicio{5} Sejam $A$ um anel e $I$ um ideal à esquerda de $A.$ Chama-se anulador de $I$ ao conjunto \[
\mbox{Ann}(I) = \{x \in
A | xm = 0, \forall m \in I \}.\] Prove que $\mbox{Ann}(I)$ é um ideal bilateral de $A.$
\solucao{Vamos mostrar que $\mbox{Ann}(I)$ é um ideal de $A.$ Temos:
\begin{itemize}
    \item $0 \in \mbox{Ann}(I),$ pois $0 \cdot m = 0,$ para todo $m \in I.$
    \item Para $a,b \in  \mbox{Ann}(I),$ temos que $am = 0$ e $bm = 0,$ para todo $m \in I.$ Logo,
    \[
    (a-b)m = am - bm = 0, \ \forall m \in I \Rightarrow a-b \in \mbox{Ann}(I).
    \]
    \item Seja $\alpha \in A.$ Para $a \in I,$ temos que $am = 0$ para todo $m \in I.$ Dessa forma,
    \[
    (\alpha a)m = \alpha \textcolor{blue}{(am)} = \alpha \cdot \textcolor{blue}{0} \Rightarrow \alpha a \in I.
    \]
    Analogamente,
\end{itemize}
Logo, $\mbox{Ann}(I)$ é um ideal bilateral de $A.$
}
\exercicio{6} Consideremos o anel $\mathcal{M}_2(\mathbb{Q})$ das matrizes $2 \times 2$ com coeficientes em $\mathbb{Q}.$
\dividiritens{
\task[\pers{a}] Prove que os único ideais bilaterais de $\mathcal{M}_2(\mathbb{Q})$ são $\{ 0\}$ e o próprio $\mathcal{M}_2(\mathbb{Q})$.
\task[\pers{b}] Dê exemplos de ideais à esquerda e à direita não triviais de $\mathcal{M}_2(\mathbb{Q})$.
\task[\pers{c}]  Generalize os resultados anteriores para um anel de matrizes $\mathcal{M}_2(F)$ onde $F$ é um corpo.
}
\solucao{
\dividiritens{
\task[\pers{a}] Seja $I \neq \{ 0 \}$ um ideal de $\mathcal{M}_2(\mathbb{Q}).$ Considere uma matriz $A \in I. $ Seja $\alpha$ uma entrada não nula de $A,$ e assuma que esta pertença à linha $\ell$ e coluna $c.$ Considere $E_1 \in \mathcal{M}_2(\mathbb{Q})$ que possui todas as suas entradas nula a menos da entrada $(1, \ell),$ que valerá $1.$ Seja $E_2 \in \mathcal{M}_2(\mathbb{Q}),$ que possui todas as suas entradas nula a menos da entrada $(c, 1),$ que valerá $1.$ Como $I$ é um ideal bilateral, temos que $B = E_1AE_2 \in I.$ Mas $B$ é a matriz que possui todas as suas entradas nulas, exceto a entrada $(1,1),$ que terá valor $\alpha.$ Para visualizar melhor, considere hipoteticamente
\[
A = \begin{pmatrix} 1 & 0 \\ \textcolor{PineGreen}{\alpha} & 3 \end{pmatrix}
\]
onde $\alpha$ pertence à linha $2$ e coluna $1.$ Nesse caso, temos que
\[
E_1 = \begin{pmatrix} 0 & \textcolor{PineGreen}{1} \\ 0 & 0 \end{pmatrix} \quad \mbox{e} \quad E_2 = \begin{pmatrix}  \textcolor{PineGreen}{1} & 0 \\ 0 & 0 \end{pmatrix}
\]
e então
\[
B = E_1AE_2 = \begin{pmatrix} 0 & \textcolor{PineGreen}{1} \\ 0 & 0 \end{pmatrix} \begin{pmatrix} 1 & 0 \\ \textcolor{PineGreen}{\alpha} & 3 \end{pmatrix} \begin{pmatrix}  \textcolor{PineGreen}{1} & 0 \\ 0 & 0 \end{pmatrix} = \begin{pmatrix} \textcolor{PineGreen}{\alpha} & 0 \\ 0 & 0 \end{pmatrix}
\]
Utilizando um argumento análogo, podemos concluir que $C \in I,$ onde $C$ é a matriz cujas entradas são nulas, exceto pela entrada $(2,2),$ que vale $\alpha,$ tomando 
\[
E_3 = \begin{pmatrix} 0 & 0 \\ 0 & \textcolor{PineGreen}{1} \end{pmatrix} \quad \mbox{e} \quad E_4 = \begin{pmatrix}  0 & \textcolor{PineGreen}{1} \\ 0 & 0 \end{pmatrix},
\]
temos
\[
C = E_3AE_4 = \begin{pmatrix} 0 & 0 \\ 0 & \textcolor{PineGreen}{1} \end{pmatrix}  \begin{pmatrix} 1 & 0 \\ \textcolor{PineGreen}{\alpha} & 3 \end{pmatrix}  \begin{pmatrix}  0 & \textcolor{PineGreen}{1} \\ 0 & 0 \end{pmatrix} = \begin{pmatrix} 0 & 0 \\ 0 & \textcolor{PineGreen}{\alpha} \end{pmatrix}
\]
Como $B, C \in I,$ e $I$ é um ideal, evidentemente $B + C \in I.$ Portanto,
\[
B + C = \begin{pmatrix} \textcolor{PineGreen}{\alpha} & 0 \\ 0 & 0 \end{pmatrix} + \begin{pmatrix} 0 & 0 \\ 0 & \textcolor{PineGreen}{\alpha} \end{pmatrix} = \begin{pmatrix} \textcolor{PineGreen}{\alpha} & 0 \\ 0 & \textcolor{PineGreen}{\alpha} \end{pmatrix} = \alpha \begin{pmatrix} 1 & 0 \\ 0 & 1 \end{pmatrix} \in I
\]
Note que $B + C$ é inversível, com inversa $(B + C)^{-1} = \begin{pmatrix} \textcolor{PineGreen}{\alpha^{-1}} & 0 \\ 0 & \textcolor{PineGreen}{\alpha^{-1}} \end{pmatrix}.$ Logo,
\[
(B + C)(B + C)^{-1} \in I \Rightarrow  \begin{pmatrix} 1 & 0 \\ 0 & 1 \end{pmatrix} \in I
\]

Como a matriz identidade está em $I,$ concluímos que $I = \mathcal{M}_2(\mathbb{Q}).$
\task[\pers{b}] 
\task[\pers{c}] Repetindo um processo análogo ao feito no item (a), podemos concluir que os únicos ideais bilaterais de $\mathcal{M}_2(F)$ são $\{ 0\}$ e o próprio $\mathcal{M}_2(F)$.
}
}
\exercicio{7} Seja $p$ um número primo. Prove que:\dividiritens{
\task[\pers{a}] Se $A$ é um anel de integridade finito de característica $p,$ então a aplicação $\varphi \colon A \to A$ dada por $\varphi(a) = a^p$ é um automorfismo de $A.$  $\varphi$ é conhecido como \textbf{automorfismo de Frobenius.}
\task[\pers{b}] O único automorfismo de $\mathbb{Z}_p$ é o automorfismo idêntico. Deduzir que a $a^p \equiv a \pmod p$ para todo $a \in A.$
\task[\pers{c}] Prove o \emph{Pequeno Teorema de Fermat}, isto é, se $p$ não divide $a,$ então $a^{p-1} \equiv 1 \pmod p.$
}
\solucao{
\dividiritens{
\task[\pers{a}] Lembrando que um automorfismo é um isomorfismo de $A$ nele mesmo, temos que:
\begin{itemize}
    \item[$\textcolor{red}{\varheart}$] $\varphi$ é homomorfismo:
\begin{itemize}
    \item  Para todos $x, y \in A,$ temos 
    \[
    \varphi(xy) = (xy)^p = \textcolor{Blue}{x^p}\textcolor{Green}{y^p} = \textcolor{Blue}{\varphi(x)}\textcolor{Green}{\varphi(y)}
    \]
    \item Vamos mostrar que, $x, y \in A,$ temos $\varphi(x+y) = \varphi(x) + \varphi(y).$ Para isso, temos pelo teorema binomial que
    \[
    \varphi(x + y) = (x+y)^p = \sum\limits_{k=0}^p \binom{p}{k} x^k y^{p-k}.
    \]
Lembrando que
\[
\binom{p}{k} = \frac{p!}{k!(p-k)!}
\]
Para $0 < k < p,$ $k!$ e $(k-p)!$ são produtos de números positivos menores do que $p.$ Consequentemente, $p$ não divide o denominador na fração acima. Como $p$ ivide o numerador, temos que $p$ divide $\binom{p}{k}.$ Mantendo em mene que $pz = 0$ para todo $z \in A,$ pois $A$ é um anel de integridade finito de característica $p,$ vemos que os termos para $0 < k< p$ do lado direito são todos nulos. Os termos remanescentes serão para $k = 0$ e $k = p,$ $x^p$ e $y^p.$ Daí,
\[
    \varphi(x + y) = (x+y)^p = \sum\limits_{k=0}^p \binom{p}{k} x^k y^{p-k} =\]\[ \textcolor{Emerald}{\binom{p}{0} x^0y^{p-0}} + \underbrace{\binom{p}{1} x^1y^{p-1} + \binom{p}{2} x^2y^{p-2} + \ldots + \binom{p}{p-1} x^{p-1}y^{p-(p-1)}}_{ = 0} + \]\[\textcolor{Magenta}{\binom{p}{p} x^py^{p-p}} = \textcolor{Emerald}{y^p} \textcolor{Magenta}{x^p} = \textcolor{Emerald}{\varphi(y)} \textcolor{Magenta}{\varphi(x)} 
\]

\item $\varphi(1) = 1^p = 1.$
\end{itemize}
   \item[$\clubsuit$] $\varphi$ é injetora: Como $\varphi(1) = 1,$ então $\varphi$ não é o homomorfismo nulo. Daí, o núcleo de $\varphi$ é um ideal $I \subseteq A,$ com $I \neq A.$ Mas como $A$ é um corpo, seus únicos ideais os  trivias. Logo, $I = \{ 0 \}.$ Portanto, $\Ker \varphi = \{ 0 \},$ e $\varphi$ é injetora.
   \item[$\spadesuit$] $\varphi$ é sobrejetora: Como $A$ é finito, e $\varphi$ é um homomorfismo injetor de $A$ em $A,$ segue que $A$ é sobrejetora.
\end{itemize}

Portanto, concluímos que $\varphi$ é um automorfismo.

\task[\pers{b}] Como $\varphi$ é um automorfismo em $\mathbb{Z}_p,$ então $\varphi$ é o automorfismo idêntico. Logo,
\[
Id(a) = a \quad \mbox{e} \quad \varphi(a) = a^p \Rightarrow \textcolor{Red}{Id} = \textcolor{Green}{\varphi} \Rightarrow  \textcolor{Red}{Id(a)} = \textcolor{Green}{\varphi(a)} \Rightarrow a = a^p
\]
Portanto, temos que
\[
a^p - a \equiv 0 \pmod p \Rightarrow \boxed{a^p \equiv a \mod p}
\]

\task[\pers{c}] Do item anterior, temos que 

\[a^p \equiv a \mod p \Rightarrow p \mid a^p - a \Rightarrow p \mid a(a^{p-1} - 1)\]

Como $p$ não divide $a,$ então $p \mid a^{p-1} - 1.$ Portanto,
\[
p \mid a^{p-1} - 1 \Rightarrow a^{p-1} - 1 \equiv 0 \pmod p \Rightarrow \boxed{a^{p-1} \equiv 1 \pmod p}
\]
}
\textbf{Observação:} Para resolver essa questão, utilizamos o seguinte resultado:
\begin{proposicao}
Todo anel de integridade finito é corpo.
\end{proposicao}
\begin{proof}
Seja $D = \{0, d_1, \ldots, d_n \}$ um anel de integridade finito. Para cada $i \in \{ 1 , 2 , \ldots , n\},$ consideremos os produtos $d_id_1, d_id_2, \ldots , d_id_n.$ Estes são distintos dois a dois, pois 
\[d_id_j = d_id_k \Leftrightarrow d_i(d_j - d_k) = 0;\] como $d_i \neq 0$ e $D$ não tem divisores de zero, necessariamente $d_j - d_k = 0,$ isto é, $d_j = d_k.$

Assim, os produtos $d_id_1, d_id_2, \ldots , d_id_n$ percorrem todos os elementos não nulos de $D;$ em particular, existe $j$ tal que $d_id_j = 1,$ o que significa que $d_i$ é invertível.

Portanto, todo o elemento não nulo de $D$ é invertível, logo $D$ é um corpo.
\end{proof}
%http://www.mat.uc.pt/~picado/algebraII/0405/Apontamentos/aula2.pdf - pg 2
}

\exercicio{8} Seja $A$ um anel tal que $x^2 = x$ para todo $x \in A,$ este anel é chamado \emph{anel de Boole}. Mostre que $A$ é um anel comutativo.
\solucao{
Seja $A$ um anel tal que $\forall x\in A:x^2=x$. Vamos primeiramente mostrar que todo elemento é igual a seu inverso. Então:
\[
\begin{array}{rcl}
x+x&=&(x+x)^2\\&=&x(x+x)+x(x+x)\\&=&\textcolor{red}{x^2}+\textcolor{Laranja}{x^2}+\textcolor{Blue}{x^2}+\textcolor{Green}{x^2}\\&=&\textcolor{red}{x}+\textcolor{Laranja}{x}+\textcolor{Blue}{x}+\textcolor{Green}{x},
\end{array}
\]
Logo,
\[
x + x = x + x +x +x \Rightarrow 0=x+x \Rightarrow x = -x
\]
Agora, note que
\[
\begin{array}{rcl}
x+y&=&(x+y)^2\\&=&x(x+y)+y(x+y)\\&=&\textcolor{CadetBlue}{x^2}+xy+yx+\textcolor{Mahogany}{y^2}\\&=&\textcolor{CadetBlue}{x}+xy+yx+\textcolor{Mahogany}{y},
\end{array}
\]
Daí:
\[
x + y = x + y + xy + yx \Rightarrow 0=xy+yx \Rightarrow -xy = yx
\]
Como todo elemento é igual a seu inverso, temos também que $xy = -xy.$ Logo:
\[
xy = -xy = yx \Rightarrow xy = yx.
\]
Portanto, o anel $A$ é comutativo.
}

\exercicio{9} Mostre que todo anel de integridade finito é um corpo. Note que $\mathbb{Z}$ é um anel de integridade infinito e não é um corpo.
\solucao{
Seja $R$ um anel de integridade finito. Tome $r \in R$ diferente de $0.$ Considere a função:
\[
\fullfunction{\varphi}{R}{R}{x}{rx}
\]
Observe que $\varphi$ é um injetora, pois se existem $x,y \in R$ tais que $\varphi(x) = \varphi(y),$ então
\[
\varphi(x) = \varphi(y) \Rightarrow rx = ry \Rightarrow r(x - y) = 0 \Rightarrow x - y = 0 \Rightarrow x = y.
\]
Sendo injetora e $R$ finito, então existe um certo $\rho \in R$ tal que $\varphi(\rho) = 1,$ ou seja, $r\rho = 1.$ Logo, $\rho$ é o inverso de $r.$ Como $r$ foi escolhido arbitrariamente, concluímos que todo elemento não nulo de $R$ é invertível. Portanto, $R$ é um corpo.
}
\exercicio{10} Sejam $A$ um anel, $I$ e $J$ ideais à direita (esquerda) de $A.$ Mostre que $I \cup J$ é um ideal à direita
(esquerda) de $A$ se e somente se $I \subset J$ ou $J \subset I.$
\solucao{Suponha por absurdo que, dados $I$ e $J$ ideais à direita (esquerda) de $A,$ $I \cup J$ é um ideal à direita (esquerda) de $A$, mas $I \nsubseteq J$ nem $J \nsubseteq I.$ Então, existem $i \in I \setminus J$ e $j \in J \setminus I.$ Sendo $I \cup J$ um ideal, então $i+j \in I \cup J.$ Então, $i+j \in I,$ o que implica $j \in I,$ e também $i+j \in J,$ o que implica $i \in J.$ Ambas conclusões são contradição, logo tais elementos $i$ e $j$ não existem.
}

\exercicio{11}  Seja $m \in \mathbb{Z}, m > 0.$ Mostre que $m\mathbb{Z}$ é um ideal maximal de $\mathbb{Z}$ se e somente se $m$ é um número primo.
\solucao{
Antes, lembremos que um ideal $I$ de um anel $R$ é dito maximal se para todo ideal $J$, se $I \subset J \subset R$, então $J = I$ ou $J = R.$ 

Suponha que $m$ é um número primo. Considere $I$ um ideal contendo $m \mathbb{Z}.$ Se $a \in I \setminus m \mathbb{Z},$ então $a$ não tem fator comum com $m,$ e portanto $\mdc(a,m) = 1.$ Pelo Teorema de Bézout, existem inteiros $r$ e $s$ tais que $as + mr = 1. $ Isso implica que $1 \in I,$ logo $I = \mathbb{Z}.$ Então $m \mathbb{Z}$ é maximal.

Se $m \mathbb{Z}$ é ideal maximal, considere $d$ um divisor de $m.$ Então $d \mathbb{Z}$ é um ideal de $\mathbb{Z},$ e $m \mathbb{Z} \subseteq d \mathbb{Z}.$ Como $m \mathbb{Z}$ é um ideal maximal, isso implica que $d \mathbb{Z} = m \mathbb{Z}$ ou $d \mathbb{Z} = \mathbb{Z}.$ Daí, $d = \pm m$ ou $d = \pm 1,$ e os únicos divisores positivos de $m$ são $1$ e $m.$ Logo, $m$ é um número primo.
}
\exercicio{12} Seja $A$ um anel tal que $x^3 = x$ para todo $x \in A.$ Mostre que $A$ é um anel comutativo.
\solucao{
Seja $A$ um anel tal que $\forall x\in A:x^3=x$. Então:
\[
\begin{array}{rcl}
2x&=&(2x)^3\\&=&8x^3\\&=&8x,
\end{array}
\]
Portanto, temos que $6x = 0.$ Além disso:
\[
\begin{array}{rcl}
x+x^2&=&(x+x^2)^3\\&=&\textcolor{red}{x^3}+3\textcolor{Laranja}{x^4}+3\textcolor{Verde}{x^5}+\textcolor{Blue}{x^6}\\&=&\textcolor{red}{x}+3\textcolor{Laranja}{x^2}+3\textcolor{Verde}{x}+\textcolor{Blue}{x^2},
\end{array}
\]
Assim, temos que
\[
x + x^2 = x + 3x^2 + 3x + x^2 \Rightarrow 3x+3x^2=0.
\]
Portanto, para todo $\alpha \in A,$ concluímos que $3\alpha + 3\alpha^2 = 0.$ Em particular, tomando $\alpha = x + y,$ temos
\[
3(x+y) + 3(x+y)^2 = 0 \Rightarrow \textcolor{Brown}{3x}+\textcolor{Purple}{3y}+\textcolor{Brown}{3x^2} + 3xy + 3yx + \textcolor{Purple}{3y^2} = 0 \Rightarrow\]\[ \textcolor{Brown}{(3x + 3x^2)} + 3xy + 3yx + \textcolor{Purple}{(3y + 3y^2)} = 0 \Rightarrow 3xy + 3yx = 0
\]
Vamos agora mostrar que $2xy - 2yx = 0,$ o que nos permitirá concluir o resultado desejado. Observe que
\[
2y = 2y + x - x  \textcolor{Fuchsia}{(x+y)} - \textcolor{Emerald}{(x-y)} \Rightarrow 2y = 2y + x - x  \Rightarrow \]\[
2y = \textcolor{Fuchsia}{(x+y)} - \textcolor{Emerald}{(x-y)} \Rightarrow 
2y = \textcolor{Fuchsia}{(x+y)^3} - \textcolor{Emerald}{(x-y)^3} \Rightarrow
2y = \]\[\textcolor{Fuchsia}{(x^3 + x^2y+xyx + xy^2 + yx^2 + yxy + y^2 x + y^3)} - \textcolor{Emerald}{(x^3 - x^2y-xyx + xy^2 - yx^2 + yxy + y^2 x - y^3)} \Rightarrow \]\[
2y = x^3 + x^2y+xyx + xy^2 + yx^2 + yxy + y^2 x + y^3 - x^3 + x^2y+xyx - xy^2 + yx^2 - yxy - y^2 x + y^3  \Rightarrow \]\[2y = 2\textcolor{Salmon}{y^3}+2x^2y+2xyx+2yx^2 \Rightarrow
\]
\[
2y = 2\textcolor{Salmon}{y}+2x^2y+2xyx+2yx^2 \Rightarrow
\]
\[
2x^2y+2xyx+2yx^2=0
\]
Ademais, multiplicando a expressão obtida acima por $x$ à esquerda e à direita, obtemos: 
\[
2x^2y+2xyx+2yx^2=0 \Rightarrow x(2x^2y+2xyx+2yx^2) = 0 \Rightarrow 2x^3y+2x^2yx+2xyx^2=0
\]
e também
\[
2x^2y+2xyx+2yx^2=0 \Rightarrow (2x^2y+2xyx+2yx^2)x = 0 \Rightarrow 2x^2yx +2xyx^2 + 2yx^3 = 0
\]
Subtraindo os resultados obtidos, chegamos a
\[
(2x^3y+2x^2yx+2xyx^2) - (2x^2yx +2xyx^2 + 2yx^3 ) = 0 \Rightarrow \]\[ 2x^3y+ \cancel{2x^2yx}+\bcancel{2xyx^2} - \cancel{2x^2yx} - \bcancel{2xyx^2} - 2yx^3 \Rightarrow 2\textcolor{WildStrawberry}{x^3}y-2y\textcolor{WildStrawberry}{x^3}=0 \Rightarrow 2\textcolor{WildStrawberry}{x}y-2y\textcolor{WildStrawberry}{x}=0
\]
Dessa forma, concluímos que 2xy - 2yx = 0. Finalmente, 
\[
(3xy - 3yx) - (2xy - 2yx) = 0 \Rightarrow xy - yx = 0 \Rightarrow \boxed{xy = yx}
\]

Portanto, temos que o anel $A$ é comutativo.
}
\exercicio{13}  Seja $A$ um anel tal que os únicos ideais à direita de $A$ são $\{0\}$ e $A.$ Mostre que $A$ é um anel com divisão ou um anel com um número primo de elementos no qual $ab = 0$ para todos $a, b \in A.$
\solucao{ Considere o conjunto
\[
I = \{ b \in R | bR = 0 \}
\]
Vamos mostrar que $I \lhd_r R,$ ou seja, que $I$ é um ideal à direita de $R.$ Temos que:
\begin{itemize}
    \item $0 \in I,$ pois $0R = 0;$
    \item Para todo $a,b \in I,$ $a - b \ in I,$ pois $(a - b)R = aR- bR = 0;$
    \item Para $r \in R$ e $a \in I,$ temos que $ar \in I,$ pois $arR = aR = 0.$
\end{itemize}

Como os únicos ideais à direita de $R$ são os triviais, temos que $I = \{0\}$ ou $I = R.$ No primeiro caso, isso significa que $ab $. Caso contrário, $I = R,$ e portanto temos que $xy = 0$ para todos $x,y \in R.$ Dessa forma, $(R, +)$ é um grupo abeliano simples, ou seja, cujos únicos subgrupos são os triviais, e portanto $\abs{R} = p$ é um número primo.
}
\exercicio{14} Sejam $F$ um corpo e $F[X,Y]$ o anel dos polinômios em duas indeterminadas com coeficientes em $F$.
Prove que $F[X, Y]$ não é um anel principal. 
\solucao{Vamos encontrar um ideal em $F[X, Y]$ que não pode ser gerado por um único elemento. Tomemos $I = \langle X, Y \rangle.$ Mostraremos que $I$ não admite um gerador único.

Suponha por absurdo que $I = \langle p \rangle,$ para certo $p(X,Y) \in F[X,Y].$ Isso quer dizer que, para todos $g(X,Y), h(X,Y) \in F[X,Y],$ existe um certo $i(X,Y) \in F[X,Y]$ tal que
\[
Xg + Yh = ip.
\]
Em particular, devemos ter $X = i_1p$ e $Y = i_2p.$ Desse modo, segue que o grau de $p$ é no máximo $1.$ Note que este grau não pode ser zero, pois caso contrário $p$ seria uma unidade ou $0$ e não teríamos $\langle p \rangle = \langle X,Y \rangle.$ Portanto, $\deg(p) = 1.$ Sendo assim, podemos escrever $p(X,Y)$ como
\[
p(X,Y) = aX + bY + c, \mbox{ onde } a,b,c \in F.
\]
Mas então
\[
X = i_1\textcolor{Magenta}{p(X,Y)} \Rightarrow X = i_1\textcolor{Magenta}{(aX + bY + c)} \Rightarrow b = 0
\]
e analogamente, 
\[
Y = i_2\textcolor{JungleGreen}{p(X,Y)} \Rightarrow X = i_1\textcolor{JungleGreen}{(aX + bY + c)} \Rightarrow a = 0
\]
Daí, concluímos que
\[
p(X,Y) = \textcolor{red}{a}X +  \textcolor{red}{b}Y + c \Rightarrow p(X,Y) = \textcolor{red}{0}X +  \textcolor{red}{0}Y + c \Rightarrow p(X,Y) = c,
\]
o que significa que $\deg(p) \neq 1,$ contradição.
Logo, $\langle X, Y \rangle$ não é um ideal principal, e portanto $F[X,Y]$ não é um anel principal.
}
\exercicio{15} Seja $A = \mathcal{C}([0, 1])$ o anel das funções contínuas de $[0, 1]$ em $\mathbb{R}.$ Se $M$ é um ideal maximal de $A$, prove que existe $c \in [0, 1]$ tal que \[M = \{f \in A | f(c) = 0\}.\]
\solucao{%http://www.math.cmu.edu/~mubayi/mathstudies/hw2sol.pdf
Para cada $c \in [0,1],$ vamos supor, por contradição, que existe $f \in M$ que não se anula em $c$ (caso contrário $M$ contém todas as funções que se anulam em $c$ e nós já teremos provado que ele é maximal, e então $M$ será o ideal desejado). Como $f$ é contínua, existe uma vizinhança $N_c$ de $c$ na qual $f(x) > y_c > 0.$ Isso produz uma coleção de conjuntos abertos que cobrem $[0,1],$ e como este é compacto, segue que existe uma subcobertura finita $T_1, \ldots, T_n.$ Também temos funções $f_1, \ldots, f_n.$ Defina aagora
\[
g(x) = \sum\limits_{i=1}^n (f_i(x))^2 \in M.
\]
Por definiçaão, existe um $\gamma > 0$ tal que $g(x) > \gamma$ para todo $x \in [0,1].$ Consequentemente, $\frac{1}{g(x)} \in M,$ pois $g(x)$ é contínua. Portanto, $1 \in M$ e $M = R.$  
}
\exercicio{16} Prove que: 
\dividiritens{
\task[\pers{a}] O ideal $I$ de $\mathbb{Z}$ é maximal se e somente se $I = \langle p \rangle,$ onde $p$ é um número primo.
\task[\pers{b}] $F[X]$ é um anel principal, onde $F$ é um corpo. Qual a condição sobre $f \in F[X]$ para que $\langle f \rangle$ seja um ideal maximal?
}
\solucao{
\dividiritens{
\task[\pers{a}] Ver Questão 11.
\task[\pers{b}] Vamos provar que $\langle f \rangle$ é um ideal maximal se e somente se $f$ é irredutível em $F[X].$

Suponha $f(x)$ irredutível. Tome $I$ um ideal de $F[X]$ tal que 
\[
\langle f \rangle \subsetneq I \subseteq F[X].
\]
Seja $g \in I \setminus \langle f \rangle.$ Podemos aplicar o Teorema de Bézout para polinômios e ver que existem $a(x)$ e $b(x)$ tais que
\[
f(x)a(x) + g(x)b(x) = 1.
\]
Portanto, $1 \in I.$ Logo, $I = F[X].$ Daí, o ideal $\langle f \rangle$ é maximal.

Reciprocamente, suponha que $\langle f \rangle$ é um ideal maximal. Tome $d(x)$ um fator de $f(x).$ então $\langle d \rangle$ é um ideal de $F[X],$ e temos $\langle f \rangle \subset \langle d \rangle.$ Isso implica que $\langle f \rangle = \langle d \rangle$ ou $\langle d \rangle = F[X],$ pois $\langle f \rangle$ é maximal. Então $d(x) \in F,$ ou $d$ é um múltiplo escalar de $f.$ Isso implica que $f$ é irredutível.
}
}
\exercicio{17} Seja $A = \mathcal{C}[0, 1]$ o anel das funções reais contínuas definidas no intervalo $[0, 1].$ Prove que 
\[I = \left\{f \in A | f \left(\frac{1}{2} \right) = 0 \right\}\]
é um ideal maximal de $A.$ 
\solucao{}
\exercicio{18} Seja $M$ um $A$-módulo. Prove que:
\dividiritens{
\task[\pers{a}] $(-a)m = a(-m) = -(am), \forall a \in A, \forall m \in M;$
\task[\pers{b}] $0 \cdot m = 0, \forall m \in M;$
\task[\pers{c}] $a \cdot 0 = 0, \forall a \in A.$
}
\solucao{
\dividiritens{
\task[\pers{a}] %Temos que
%\[(0 + (-a))m = 0 \cdot m + (-a) \cdot m \Rightarrow (0 -a )m = 0 \cdot m -a \cdot m \Rightarrow (-a)m = -(am)\]
%e também:
%\[a(0 + (-m)) = a \cdot \]
\task[\pers{b}] Temos que, $\forall m \in M:$
\[
0 \cdot m = (0+0)m \Rightarrow 0 \cdot m + 0 \cdot m = 0 \cdot m \Rightarrow 0 \cdot m = 0
\]
Note que utilizamos que $(a+b)m = am + bm, \forall a,b \in A, \ \forall m \in M.$
\task[\pers{c}] Temos que, $\forall a \in A:$
\[
a \cdot 0 = a(0+0) \Rightarrow a \cdot 0 = a \cdot 0 + a \cdot 0  \Rightarrow a \cdot 0 = 0
\]
Note que utilizamos que $a(m+n) = am + an, \forall a \in A, \ \forall m,n \in M.$
}
}
\exercicio{19}  Sejam $M$ um $A$-módulo e $S$ e $T$ submódulos de $M.$ Prove que $S \cup T$ é um submódulo de $M$ se e somente se $S \subset T$ ou $T \subset S.$ 
\solucao{Suponha por absurdo que, dados $S$ e $T$ ideais à direita (esquerda) de $M,$ $S \cup T$ é um ideal à direita (esquerda) de $M$, mas $S \nsubseteq T$ nem $T \nsubseteq S.$ Então, existem $s \in S \setminus T$ e $t \in T \setminus S.$ Sendo $S \cup T$ um submódulo à esquerda (direita), então $s+t \in S \cup T.$ Então, $s+t \in S,$ o que implica $t \in S,$ e também $s+t \in T,$ o que implica $s \in T.$ Ambas conclusões são contradição, logo tais elementos $s$ e $t$ não existem.}
\exercicio{20} Determinar todos os submódulos do $\mathbb{Z}$-módulo $\mathbb{Z}_{12},$ o anulador de cada elemento de $\mathbb{Z}_{12}$ e o anulador do módulo todo.
\solucao{
Notemos que os $\mathbb{Z}$-submódulos de $\mathbb{Z}_{12}$ são exatamente os subgrupos de $\mathbb{Z}_{12},$ já que este é abeliano. Como $\langle \overline{1} \rangle = \mathbb{Z}_{12}$ é cíclico, todos os seus subgrupos são cíclicos. Então, basta analisar o subgrupo gerado por cada elemento de $\mathbb{Z}_{12}.$ Os possíveis $\mathbb{Z}$-submódulos de $\mathbb{Z}_{12}$ são:
\begin{itemize}
    \item $ \langle \overline{0} \rangle = \{ \overline{0} \};$
    \item $ \langle \overline{1} \rangle = \{ \overline{0}, \overline{1}, \overline{2},\overline{3}, \overline{4}, \overline{5}, \overline{6}, \overline{7}, \overline{8}, \overline{9}, \overline{10}, \overline{11}\} = \mathbb{Z}_{12};$
    \item $ \langle \overline{2} \rangle = \{ \overline{0},  \overline{2}, \overline{4},  \overline{6}, \overline{8},  \overline{10}\};$
    \item $ \langle \overline{3} \rangle = \{ \overline{0}, \overline{3}, \overline{6}, \overline{9}\};$
    \item $ \langle \overline{4} \rangle = \{ \overline{0}, \overline{4}, \overline{8}\} ;$
    \item $ \langle \overline{6} \rangle = \{ \overline{0},  \overline{6}\}.$
\end{itemize}

Logo, $\mathbb{Z}_{12}$ possui $6$ $\mathbb{Z}$-submódulos.

Calculemos o anulador de cada elemento de $\mathbb{Z}_{12}:$
\[
\mbox{Ann}(\overline{0}) = \{a \in \mathbb{Z}_{12} | a \cdot \overline{0} = \overline{0} \} = \mathbb{Z}_{12};
\]
\[
\mbox{Ann}(\overline{1}) = \{a \in \mathbb{Z}_{12} | a \cdot \overline{1} = \overline{0} \} = \{ \overline{0} \} = \langle \overline{0} \rangle;
\]
\[
\mbox{Ann}(\overline{2}) = \{a \in \mathbb{Z}_{12} | a \cdot \overline{2} = \overline{0} \} = \{ \overline{0}, \overline{6} \} = \langle \overline{6} \rangle;
\]
\[
\mbox{Ann}(\overline{3}) = \{a \in \mathbb{Z}_{12} | a \cdot \overline{3} = \overline{0} \} = \{ \overline{0}, \overline{4}, \overline{8} \} = \langle \overline{4} \rangle;
\]
\[
\mbox{Ann}(\overline{4}) = \{a \in \mathbb{Z}_{12} | a \cdot \overline{4} = \overline{0} \} = \{ \overline{0}, \overline{3}, \overline{6}, \overline{9} \} = \langle \overline{3} \rangle;
\]
\[
\mbox{Ann}(\overline{5}) = \{a \in \mathbb{Z}_{12} | a \cdot \overline{5} = \overline{0} \} = \{ \overline{0} \} = \langle \overline{0} \rangle;
\]
\[
\mbox{Ann}(\overline{6}) = \{a \in \mathbb{Z}_{12} | a \cdot \overline{6} = \overline{0} \} = \{ \overline{0}, \overline{2}, \overline{4}, \overline{6}, \overline{8}, \overline{10} \} = \langle \overline{2} \rangle;
\]
\[
\mbox{Ann}(\overline{7}) = \{a \in \mathbb{Z}_{12} | a \cdot \overline{7} = \overline{0} \} = \{ \overline{0} \} = \langle \overline{0} \rangle;
\]
\[
\mbox{Ann}(\overline{8}) = \{a \in \mathbb{Z}_{12} | a \cdot \overline{8} = \overline{0} \} = \{ \overline{0}, \overline{3}, \overline{6}, \overline{9}\} = \langle \overline{3} \rangle;
\]
\[
\mbox{Ann}(\overline{9}) = \{a \in \mathbb{Z}_{12} | a \cdot \overline{9} = \overline{0} \} = \{ \overline{0}, \overline{4}, \overline{8}\} = \langle \overline{4} \rangle;
\]
\[
\mbox{Ann}(\overline{10}) = \{a \in \mathbb{Z}_{12} | a \cdot \overline{10} = \overline{0} \} = \{ \overline{0}, \overline{6}\} = \langle \overline{6} \rangle;
\]
\[
\mbox{Ann}(\overline{11}) = \{a \in \mathbb{Z}_{12} | a \cdot \overline{11} = \overline{0} \} = \{ \overline{0} \} = \langle \overline{0} \rangle.
\]
Temos portanto os seguintes resultados:
 \begin{table}[h]
 \begin{minipage}{.5\textwidth}
 \centering
 \begin{tabular}{|c|c|}
 \toprule
 Elemento & Anulador \\ \hline
$\overline{0}$ & $\mathbb{Z}_{12}$ \\\hline
$\overline{1}$ & $\langle \overline{0} \rangle$ \\\hline
$\overline{2}$ & $\langle \overline{6} \rangle$ \\\hline
$\overline{3}$ & $\langle \overline{4} \rangle$ \\\hline
$\overline{4}$ & $\langle \overline{3} \rangle$ \\\hline
$\overline{5}$ & $\langle \overline{0} \rangle$ \\ \hline
 \end{tabular}
  
 \end{minipage} 
 \begin{minipage}{.5\textwidth}
 \centering
 \begin{tabular}{|c|c|}
 \toprule
 Elemento & Anulador \\ \hline
$\overline{6}$ & $\langle \overline{2} \rangle$ \\\hline
$\overline{7}$ & $\langle \overline{0} \rangle$ \\\hline
$\overline{8}$ & $\langle \overline{3} \rangle$ \\\hline
$\overline{9}$ & $\langle \overline{4} \rangle$ \\\hline
$\overline{10}$ & $\langle \overline{6} \rangle$ \\\hline
$\overline{11}$ & $\langle \overline{0} \rangle$ \\ \hline
 \end{tabular}
  
 \end{minipage} 
\end{table}

Temos também que
\[
\mbox{Ann}(\mathbb{Z}_{12}) = \{ a \in \mathbb{Z}_{12} | ax = \overline{0}, \ \forall x \in \mathbb{Z}_{12} \} = \{ 0 \}
\]
}
\exercicio{21} Dar um exemplo de um $\mathbb{Z}$-módulo, onde dois submódulos quaisquer sejam sempre não isomorfos.
\solucao{Considere o $\mathbb{Z}$-módulo $\mathbb{Z}_{12},$ dado no exercício acima. Observe que a cardinalidade de todos os seus submódulos são distintas, de modo que estes não podem ser $2$ a $2$ isomorfos.}
\exercicio{22} Prove que se $m$ e $n$ são dois inteiros relativamente primos, o único $\mathbb{Z}$-homomorfismo $\varphi \colon \mathbb{Z}_n \to \mathbb{Z}_m$ é
o homomorfismo nulo.
\solucao{
Vamos mostrar algo mais forte: a quantidade de $\mathbb{Z}$-homomorfismo $\varphi \colon \mathbb{Z}_n \to \mathbb{Z}_m$ é exatamente $\mdc(m,n).$ 

Como $1$ é gerador de $\mathbb{Z}_n,$ podemos escolher um certo $\overline{a} \in \mathbb{Z}_m,$ onde $0 \le a \le m-1$ tal que $\varphi(\ovelrine{1}) = \overline{a}.$ Sendo um homomorfismo, para qualquer $n \in \mathbb{N},$ temos que
\[
\varphi(n) = \varphi(\underbrace{\ovelrine{1} + \ovelrine{1} + \ldots + \ovelrine{1}}_{n \mbox{ vezes}}) = \underbrace{\varphi(\ovelrine{1}) + \varphi(\ovelrine{1}) + \ldots + \varphi(\ovelrine{1})}_{n \mbox{ vezes}} =n \textcolor{RawSienna}{\varphi(\overline{1})} =n \textcolor{RawSienna}{\overline{a}} = \overline{na}
\]
Como $\overline{na} \in \mathbb{Z}_m,$ temos que $na \equiv 0 \pmod m.$ Daí, temos que $m \mid na.$ Como $\mdc(m,n) = 1,$ temos que $m \nmid n.$ Logo, $m \mid a.$ Sendo $0 \le a \le m-1,$ concluímos que $a = 0.$ Daí, $\varphi(\overline{1}) = \overline{a} = \overline{0},$ e portanto $\varphi$ é o homomorfismo nulo.



%https://math.stackexchange.com/questions/1807358/number-of-homomorphisms-between-two-cyclic-groups


}
\exercicio{23} Seja $M$ um $\mathbb{Z}$-módulo finito tal que o conjunto dos seus submódulos é totalmente ordenado por
inclusão. Prove que existe um número primo $p$ tal que o número de elementos de $M$ é uma potência de $p.$
\solucao{
Se a ordem de $M$ não for uma potência de um primo, então existem dois números primos $r$ e $s$ distintos que dividem a ordem de $M.$ Pelo Teorema de Cauchy, como $(M, +)$ é um grupo abeliano finito, existem subgrupos $R$ e $S$ tais que $\abs{R} = r$ e $\abs{S} = s.$ Sendo $R$ e $S$ de ordem prima e abelianos, então $R$ e $S$ são cíclicos. Como o conjunto de submódulos de $M$ é totalmente ordenado por inclusão, então $S \subset R$ ou $R \subset S,$ o que implica $s < r$ ou $r < s.$ Suponha sem perda de generalidade que $S \subset R.$ 

Pelo Teorema de Lagrange, sabemos que a ordem de todo subgrupo de $R$ divide a ordem de $R.$ Sendo $S \subset R,$ então 

\[\textcolor{Green}{\abs{S}} \mid \textcolor{Green}{\abs{R}} \Rightarrow \textcolor{Green}{s} \mid \textcolor{Green}{r},\]
um absurdo, já que $r$ e $s$ são primos.

Logo, a ordem de $M$ é uma potência de $p.$
}
\exercicio{24} Um $A$-módulo $M$ é \emph{simples} se $M \neq \{0\}$ e os únicos submódulos de $M$ são $\{ 0\}$ e o próprio $M.$
\dividiritens{
\task[\pers{a}] Prove que se $M$ é simples e $\varphi \colon M \to N$ é um $A$-homomorfismo não nulo, então $\varphi$ é um
monomorfismo. Prove que se $N$ também é simples, então $\varphi$ é um isomorfismo.
\task[\pers{b}]  Seja $\mbox{Hom}(M, N)$ o conjunto de todos os $A$-homomorfismos de $M$ em $N.$ Mostrar que com a
soma definida pontualmente e o produto por composição, $\mbox{Hom}(M, M)$ é um anel. Prove que se $M$ é simples, então $\mbox{Hom}(M, M)$ é um anel com divisão. (este resultado é conhecido como
\emph{Lema de Schur})
}
\solucao{
\dividiritens{
\task[\pers{a}] Sabemos que um $A$-homomorfismo de módulos $\varphi$ é injetor, isto é, é um monomorfismo, se e somente se $\Ker(\varphi) = \{ 0 \}.$ Também sabemos que $\Ker(\varphi)$ é um $A$-submódulo de $M.$ Sendo $\varphi \colon M \to N$ um $A$-homomorfismo, então como $\varphi \ncong 0,$ temos que $\Ker(\varphi)$ é um submódulo de $M$ tal que $\Ker(\varphi) \neq M.$ Sendo $M$ simples, seus únicos submódulos são $\{ 0\}$ e o próprio $M.$ Mas $\Ker(\varphi) \neq M.$ Logo, segue que a única possibilidade é $\Ker(\varphi) = \{ 0 \}.$ Portanto, $\varphi$ é um monomorfismo.

Sabemos que $\mbox{Im}(\varphi)$ é um submódulo de $N.$ Se $N$ for simples, então as únicas possibilidades para $\mbox{Im}(\varphi)$ são $\{ 0 \}$ ou $N.$ Mas note que $\mbox{Im}(\varphi) \neq \{ 0 \},$ pois caso contrário $\varphi$ seri o homomorfismo nulo. Logo, $\mbox{Im}(\varphi) = N,$ e $\varphi$ é sobrejetora. Sendo injetora também, segue que $\varphi$ é um isomorfismo.

\task[\pers{b}] Para provar que $\mbox{Hom}(M, M)$ é um anel com divisão, precisamos verificar que todo elemento não-nulo de $\mbox{Hom}(M, M)$ é invertível. Para isso, tome $\varphi \in \mbox{Hom}(M, M)$ não nulo. Como $M$ é simples, então $\varphi \colon M \to M$ é um $A$-homomorfismo não nulo, então pelo item a,$\varphi$ é um isomorfismo. Consequentemente, este possui uma inversa. Portanto,  $\mbox{Hom}(M, M)$ é um anel com divisão.
}


}
\exercicio{25} Prove que, se a sequência $M \xrightarrow{\varphi} N \xrightarrow{\psi} R \xrightarrow{\phi} S$ é exata, são equivalentes:
\dividiritens{
\task[\pers{a}] $\varphi$ é epimorfismo;
\task[\pers{b}] $\mbox{Im}(\psi) = 0;$
\task[\pers{c}] $\phi$ é monomorfismo.
}
\solucao{
Primeiramente, observemos que, como a sequência é exata, temos que
\[
\mbox{Im } \varphi = \Ker \psi \quad \mbox{e} \quad \mbox{Im } \psi = \Ker \phi.
\]
Com base nisso, temo que:

$\textbf{\textcolor{Floresta}{(a)}} \Rightarrow \textbf{\textcolor{Floresta}{(b)}}:$ Se $\varphi$ é epimorfismo, então $\mbox{Im}(\varphi)=N.$ Então \[\mbox{Ker}(\psi)=\mbox{Im}(\varphi)=N,\] aí $\mbox{Im}(\psi)=0$.
\medskip
\noindent
$\textbf{\textcolor{Floresta}{(b)}} \Rightarrow \textbf{\textcolor{Floresta}{(c)}}:$
Se $\mbox{Im}(\psi)=0$, então temos \[\mbox{Ker}(\phi)=\mbox{Im}(\psi)=0.\]
Logo,$\phi$ é monomorfismo.
\medskip
\noindent
$\textbf{\textcolor{Floresta}{(c)}} \Rightarrow \textbf{\textcolor{Floresta}{(a)}}:$ Se $\phi$ é monomorfismo, então $\Ker \phi = 0.$ Pela sequência exata, temos
\[\mbox{Im}(\psi)=\mbox{Ker}(\phi)=0 \Rightarrow \mbox{Ker}(\psi)=N.\]
Então, \[\mbox{Im}(\varphi)=\mbox{Ker}(\psi)=N.\] 

Assim, $\varphi$ é epimorfismo.
}
\exercicio{26} Seja o diagrama comutativo:
\begin{center}
\begin{tikzcd}
            & M^{\prime} \arrow{r}{f^{\prime}} \arrow{d}{\varphi^{\prime}} & M \arrow{r}{f} \arrow{d}{\varphi} & M^{\prime \prime} \arrow{r} \arrow{d}{\varphi^{\prime \prime}} & 0 \\
0 \arrow{r} & N^{\prime} \arrow{r}{g^{\prime}}                               & N \arrow{r}{g}                      & N^{\prime \prime}                                                 &  
\end{tikzcd}
\end{center}

Suponhamos que as filas são sequências exatas, prove que
\dividiritens{
\task[\pers{a}] Se $\varphi^\prime$ e $\varphi^{\prime \prime}$ são epimorfismos, então $\varphi$ é epimorfismo.
\task[\pers{b}] Se $\varphi^\prime$ e $\varphi^{\prime \prime}$ são isomorfismos, então $\varphi$ é isomorfismo.
}
\solucao{
\dividiritens{
\task[\pers{a}] Se $\varphi^\prime$ e $\varphi^{\prime \prime}$ são epimorfismos, então para $n\in N$ então $g(n)\in N''$, aí existe um $m''\in M''$ tal que $\varphi''(m'')=g(n)$, aí existe $m\in M$ tal que $f(m)=m''$, aí $g(\varphi(m))=\varphi''(f(m))=\varphi''(m'')=g(n)$, aí $n-\varphi(m)\in\mathrm{Ker}(g)=\mathrm{Im}(g')$, aí existe $n'\in N'$ tal que $g'(n')=n-\varphi(m)$, aí existe $m'\in M'$ tal que $\varphi'(m')=n'$, aí $\varphi(f'(m'))=g'(\varphi'(m'))=g'(n')=n-\varphi(m)$, aí $n=\varphi(m+f'(m'))$, aí $n\in\mathrm{Im}(\varphi)$; assim $\varphi$ é epimorfismo.
\task[\pers{b}] Se $\varphi^\prime$ e $\varphi^{\prime \prime}$ são monomorfismos, então para $m\in M$, se $\varphi(m)=0$, então $\varphi''(f(m))=g(\varphi(m))=g(0)=0$, aí $f(m)=0$, aí $m\in\mathrm{Ker}(f)=\mathrm{Im}(f')$, aí existe $m'\in M'$ tal que $f'(m')=m$, aí $g'(\varphi'(m'))=\varphi(f'(m'))=\varphi(m)=0$, aí $\varphi'(m')=0$, aí $m'=0$, aí $f'(m')=0$, aí $m=0$; logo $\varphi$ é monomorfismo. Agora é só juntar com o que fizemos no item (a).
}
}
\exercicio{27} Seja o diagrama comutativo:
\begin{center}
\begin{tikzcd}
M_1 \arrow{r}{f_1} \arrow{d}{h_1} & M_2 \arrow{r}{f_2} \arrow{d}{h_2} & M_3 \arrow{r}{f_3} \arrow{d}{h_3} & M_4 \arrow{r}{f_4} \arrow{d}{h_4} & M_5 \arrow{d}{h_5} \\
N_1 \arrow{r}{g_1}                   & N_2 \arrow{r}{g_2}                   & N_3 \arrow{r}{g_3}                   & N_4 \arrow{r}{g_4}                   & N_5                  
\end{tikzcd}
\end{center}
Suponhamos que as filas são sequências exatas, prove que
\dividiritens{
\task[\pers{a}] Se $h_1$ é epimorfismo e $h_4$ é monomorfismo, então $\ker(h_3) = f_2(\ker(h_2));$
\task[\pers{b}] Se $h_2$ é epimorfismo e $h_5$ é monomorfismo, então $g_3^{-1}(\mbox{Im}(h_4)) = \mbox{Im}(h_3);$
\task[\pers{c}] (Lema dos Cinco) Se $h_1, h_2, h_4$ e $h_5$ são isomorfismos, então $h_3$ é um isomorfismo.
}


\solucao{
\dividiritens{
\task[\pers{a}] Vamos mostrar que $\ker(h_3) \subseteq f_2(\ker(h_2))$ e que $f_2(\ker(h_2)) \subseteq \ker(h_3):$
\begin{itemize}
    \item[$\textcolor{red}{\vardiamond}$] $f_2(\ker(h_2)) \subseteq \ker(h_3):$ Seja $m_2\in M_2$ tal que $h_2(m_2)=0$, então $g_2(h_2(m_2))=0$, aí $h_3(f_2(m_2))=0$. 
    
    \item[$\spadesuit$] $\ker(h_3) \subseteq f_2(\ker(h_2)):$ Seja $m_3\in M_3$ tal que $h_3(m_3)=0$, então $g_3(h_3(m_3))=0$, aí $h_4(f_3(m_3))=0$, aí $f_3(m_3)=0$, aí existe $m_2\in M_2$ tal que $f_2(m_2)=m_3$, aí $g_2(h_2(m_2))=h_3(f_2(m_2))=h_3(m_3)=0$, aí existe $n_1\in N_1$ tal que $g_1(n_1)=h_2(m_2)$, aí existe $m_1\in M_1$ tal que $h_1(m_1)=n_1$, aí $h_2(f_1(m_1))=g_1(h_1(m_1))=g_1(n_1)=h_2(m_2)$, aí $h_2(m_2-f_1(m_1))=0$, mas nós temos $f_2(m_2-f_1(m_1))=f_2(m_2)-f_2(f_1(m_1))=m_3$.
\end{itemize}
\task[\pers{b}] Vamos mostrar que $g_3^{-1}(\mbox{Im}(h_4)) \subseteq \mbox{Im}(h_3)$ e que $\mbox{Im}(h_3) \subseteq g_3^{-1}(\mbox{Im}(h_4)):$
\begin{itemize}
    \item[$\textcolor{red}{\varheart}$] $\mbox{Im}(h_3) \subseteq g_3^{-1}(\mbox{Im}(h_4)):$ Seja $x_3\in M_3$, então $g_3(h_3(x_3))=h_4(f_3(x_3))\in\mathrm{Im}(h_4)$, aí $h_3(x_3)\in g_3^{-1}(\mbox{Im}(h_4))$.
    
    \item[$\clubsuit$] $g_3^{-1}(\mbox{Im}(h_4)) \subseteq \mbox{Im}(h_3):$ Seja $n_3\in g_3^{-1}(\mbox{Im}(h_4))$, então existe $m_4\in M_4$ tal que $g_3(n_3)=h_4(m_4)$, aí $h_5(f_4(m_4))=g_4(h_4(m_4))=g_4(g_3(n_3))=0$, aí $f_4(m_4)=0$, aí existe $m_3\in M_3$ tal que $f_3(m_3)=m_4$, aí $g_3(h_3(m_3))=h_4(f_3(m_3))=h_4(m_4)=g_3(n_3)$, aí $g_3(n_3-h_3(m_3))=0$, aí existe $n_2\in N_2$ tal que $g_2(n_2)=n_3-h_3(m_3)$, aí existe $m_2\in M_2$ tal que $h_2(m_2)=n_2$, aí $h_3(f_2(m_2))=g_2(h_2(m_2))=g_2(n_2)=n_3-h_3(m_3)$, aí $n_3=h_3(m_3+f_2(m_2))$, aí $n_3\in\mathrm{Im}(h_3)$.
\end{itemize}
\task[\pers{c}] Se $h_1, h_2, h_4$ e $h_5$ são isomorfismos, então, pelos itens (a) e (b), nós obtemos $\ker(h_3) = f_2(\ker(h_2))=f_2(0)=0$ e $\mbox{Im}(h_3) = g_3^{-1}(\mbox{Im}(h_4))=g_3^{-1}(N_4)=N_3$, assim $h_3$ é isomorfismo.
}
%http://www.rc.unesp.br/igce/matematica/bicmat/volume_10.pdf - pg 9
}

\textcolor{Red}{Questões Suplementares}

\exercicio{28} Prove que para todo número primo $p$, existe um anel não comutativo sem unidade com exatamente $p^2$ elementos.
\solucao{Para $p$ primo, temos que
\[
\mathbb{Z}_p = \{ \overline{0}, \overline{1}, \overline{2}, \ldots, \overline{p-1} \}
\]
é um corpo de ordem $p$. Sob adição e multiplicação módulo $p,$ $\mathcal{M}_2(\mathbb{Z}_p)$ é o anel das matrizes de ordem $2 \times 2$ cujas entradas são elementos de $\mathbb{Z}_p.$
Considere
\[
R = \left\{ \begin{bmatrix} a & b \\ 0 & 0 \\ 
\end{bmatrix} \Big| a,b \in \mathbb{Z}_p \right\},
\]
um subconjunto não vazio finito de $\mathcal{M}_2(\mathbb{Z}_p)$.
Para cada $X = \begin{bmatrix} a & b \\ 0 & 0 \\ 
\end{bmatrix}$ e $Y = \begin{bmatrix} c & d \\ 0 & 0 \\ 
\end{bmatrix},$ temos que
\begin{itemize}
    \item $X + Y = \begin{bmatrix} a & b \\ 0 & 0 \\ 
\end{bmatrix} +  \begin{bmatrix} c & d \\ 0 & 0 \\ 
\end{bmatrix} =  \begin{bmatrix} a+c & b+d \\ 0 & 0 \\ 
\end{bmatrix} \in R;$
\item $X \cdot Y = \begin{bmatrix} a & b \\ 0 & 0 \\ 
\end{bmatrix} \cdot  \begin{bmatrix} c & d \\ 0 & 0 \\ 
\end{bmatrix} = \begin{bmatrix} ac & ad \\ 0 & 0 \\ 
\end{bmatrix} \in R.$
\end{itemize}
Portanto, $(R, +, \cdot)$ é um anel (subanel de $\mathcal{M}_2(\mathbb{Z}_p).$ É um anel que possui identidade à esquerda $\begin{bmatrix} 1 & 0 \\ 0 & 0 \end{bmatrix},$ mas não possui identidade à direita. Além disso, $\begin{bmatrix} 1 & 0 \\ 0 & 0 \end{bmatrix}, \begin{bmatrix} 0 & 1 \\ 0 & 0 \end{bmatrix} \in R.$ Então
\[\begin{bmatrix} 1 & 0 \\ 0 & 0 \end{bmatrix} \cdot \begin{bmatrix} 0 & 1 \\ 0 & 0 \end{bmatrix} = \begin{bmatrix} 0 & 1 \\ 0 & 0 \end{bmatrix} \neq \begin{bmatrix} 0 & 0 \\ 0 & 0 \end{bmatrix} =   \begin{bmatrix} 0 & 1 \\ 0 & 0 \end{bmatrix} \begin{bmatrix} 1 & 0 \\ 0 & 0 \end{bmatrix}
\]
Logo, concluímos que $R$ é umm anel não comutativo de ordem $p^2$ que não possui unidade.%http://marathwadamathsociety.org/vol13-1/june12salunke3%20pg%2039-47.pdf
}
\newpage
\section{\textcolor{Floresta}{Lista 2}}

\exercicio{1} Sejam $\{ M_i\}_{i \in I}$ e $\{ N_i\}_{i \in I}$ duas famílias de $A$-módulos. Se $N_i$ é submódulo de $M_i$ para todo $i \in I,$ prove que $\bigoplus\limits_{i \in I} N_i$ pode ser identificado naturalmente como um submódulo de $\bigoplus\limits_{i \in I} M_i$ e   \[\frac{\bigoplus\limits_{i \in I} M_i}{\bigoplus\limits_{i \in I} N_i } \cong \bigoplus\limits_{i \in I} \frac{M_i}{N_i}.\] 
\solucao{}

\exercicio{2} Sejam $M$ um $A$-módulo, $N_1$ e $N_2$ submódulos de $M$ tais que $M = N_1 \oplus N_2.$ Seja $N^\prime_1$ um submódulo de $M$ isomorfo a $N_1.$ Em geral, não é verdade que
\[
\frac{N_1 \oplus N_2}{N_1^{\prime}} \cong N_2
\]
Dê um contra-exemplo.
\solucao{}

\exercicio{3} Sejam $F$ um corpo e $A = \mathcal{M}_n(F).$ Prove que:
\dividiritens{
\task[\pers{a}] Os subconjuntos
\[
S_k = \{ (a_{ij}) \in A | a_{ij} = 0 \ \mbox{se} \ i \neq k \}
\]
são submódulos de ${}_AA;$
\task[\pers{b}] Os submódulos $S_k$, para $1 \le k \le n,$ são simples, isto é, os únicos submódulos dele são $\{0\}$ e ele próprio;
\task[\pers{c}] A = $S_1 \oplus S_2 \oplus \ldots \oplus S_n.$
}
\solucao{}

\exercicio{4} Dada uma sequência exata
\begin{center}
\begin{tikzcd}
0 \arrow{r} & M \arrow{r}{f} & N \arrow{r}{g} & P \arrow{r} & 0,
\end{tikzcd}
\end{center}
nem sempre $N^{\prime} = \mbox{Im}(f)$ é um somando direto de $N.$ Dê um caontra-exemplo.
\solucao{}

\exercicio{5} Dadas as sequências exatas de $A$-módulos:
\begin{center}
\begin{tikzcd}
0 \arrow{rd} &                     &   &                      & 0 \arrow{ld} \\
             & M_1 \arrow{rd}{f} &   & L_2 \arrow[swap]{ld}{k} &              \\
             &                     & N &                      &              \\
             & M_2 \arrow{ru}{h} &   & L_1 \arrow[swap]{lu}{g} &              \\
0 \arrow{ru} &                     &   &                      & 0 \arrow{lu}
\end{tikzcd}
\end{center}
Prove que, se $k \circ f\colon M_1 \to L_2$ é um isomorfismo, então $g \circ h \colon  M2 \to L_1$ também o é.
\solucao{}

\exercicio{6} Sejam $A$ um anel de integridade, $M$ um $A$-módulo livre, $a \in A$ e $m \in M.$ Prove que, se $am = 0,$
então $a = 0$ ou $m = 0.$
\solucao{}

\exercicio{7} Verificar que o $\mathbb{Z}$-módulo $\mathbb{Q}$ não é um $\mathbb{Z}$-módulo livre, apesar de ter a propriedade acima.
\solucao{}

\exercicio{8} Provar que, se $\{M_i\}_{i\in I}$ é uma família de $A$-módulos livres, então 
\[\bigoplus\limits_{i \in I} M_i\]
é livre. O produto
\[\prod\limits_{i \in I} M_i\]
é livre?
\solucao{}

\exercicio{9} Dar um exemplo para mostrar que, se $N$ é um submódulo livre de um $A$-módulo livre $M,$ nem sempre é verdade que $M/N$ é livre.
\solucao{}

\exercicio{10} Provar que não existem:
\dividiritens{ 
\task[\pers{a}] $\mathbb{Z}$-isomorfismos $f \colon {}_\mathbb{Z}\mathbb{Z} \to \mathbb{Q}.$
\task[\pers{b}]  epimorfismos de $\mathbb{Q}$ sobre $\mathbb{Z}$-módulos livres, não nulos.
}
\solucao{}
\exercicio{11} Sejam $A$ um domínio de integridade e $K$ o seu corpo de frações. Mostrar que $K$ é um $A$-módulo e que um submódulo de $K$ é livre se e somente se é cíclico.
\solucao{}

\exercicio{12} Seja $A$ um anel de integridade. Mostre que $M$ é um $A$-módulo livre de posto $n$ se e somente se $M \cong A^n.$
\solucao{}


\exercicio{13} Sejam $A$ um anel de integridade e $N$ um submódulo de um $A$-módulo $M.$ Mostre que $T(N) =
T(M) \cap N.$

\solucao{}

\exercicio{14} Provar que soma direta de módulos de torção é um módulo de torção. Dar um contra-exemplo para mostrar que o enunciado similar não é válido, em geral, para produtos diretos.
\solucao{}

\exercicio{15} Prove que:
\dividiritens{
\task[\pers{a}] Um $A$-módulo $M$ é cíclico se e somente se é isomorfo a um quociente da forma $A/I,$ onde $I$ é um ideal à esquerda de $A;$
\task[\pers{b}] Se $A$ é um anel de integridade, então $A$ é um corpo se e somente se todo $A$-módulo é sem
torção.}

\solucao{}
\end{document} 
https://math.stackexchange.com/questions/1617301/number-of-homomorphisms-from-mathbbz-m-to-mathbbz-n?noredirect=1&lq=1
https://math.stackexchange.com/questions/798119/quotient-modules-isomorphic-rightarrow-submodules-isomorphic


